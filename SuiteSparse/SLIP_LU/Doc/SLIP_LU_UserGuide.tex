\documentclass[12pt]{article}
\usepackage{amsmath,amssymb,amsthm,latexsym,paralist,comment}
\usepackage{graphicx}
\usepackage{psfrag}
\usepackage{amsmath}
\usepackage{multirow}
\usepackage{algorithm}
\usepackage{algpseudocode}
\usepackage{cprotect}
\usepackage{graphicx}
\usepackage{subcaption}
\usepackage{listings}
\usepackage[dvipsnames]{xcolor}
\usepackage[hidelinks]{hyperref}
\usepackage{framed}
\usepackage{mdframed}
\hypersetup{
       colorlinks = true,
       citecolor = blue,
       linkcolor = blue,
       urlcolor = Maroon
}
\usepackage{geometry}
\theoremstyle{definition}

\newcommand{\N}{\mathbf{N}}
\newcommand{\R}{\mathbf{R}}
\newcommand{\Z}{\mathbf{Z}}
\newcommand{\import}{\textcolor{red}{\textbf{**IMPORTANT**}}}

\begin{document}

\begin{center}
\begin{large}
\textbf{User Guide for SLIP LU, A Sparse Left-Looking Integer
Preserving LU Factorization} \\
\vspace{5mm}
Version 1.0.2, July 14, 2020 % VERSION
\vspace{20mm}

Christopher Lourenco, Jinhao Chen, \\ Erick Moreno-Centeno, Timothy A. Davis \\

Texas A\&M University

\vspace{20mm}
Contact Information: Contact Chris Lourenco, \href{mailto:chrisjlourenco@gmail.com}{chrisjlourenco@gmail.com}, or Tim Davis,
\href{mailto:timdavis@aldenmath.com}{timdavis@aldenmath.com},
\href{mailto:davis@tamu.edu}{davis@tamu.edu},
\href{DrTimothyAldenDavis@gmail.com}{DrTimothyAldenDavis@gmail.com}

\end{large}
\end{center}

\newpage
\tableofcontents

\newpage

%-------------------------------------------------------------------------------
\section{Overview}
\label{s:intro}
%-------------------------------------------------------------------------------

SLIP LU is a software package designed to exactly solve unsymmetric sparse
linear systems, $ A x = b$, where $A \in \mathbb{Q}^{n \times
n}$, $b \in \mathbb{Q}^{n \times r}$, and $x \in \mathbb{Q}^{n \times
r}$. This package performs a left-looking, roundoff-error-free (REF) LU
factorization $P A Q = L D U$, where $L$ and $U$ are integer, $D$ is diagonal,
and $P$ and $Q$ are row and column permutations, respectively. 
Note that the matrix $D$ is never explicitly computed nor needed; thus this 
package uses only the matrices $L$ and $U$. The theory associated with this code 
is the Sparse Left-looking Integer-Preserving (SLIP) LU factorization
 \cite{lourenco2019exact}. Aside from
solving sparse linear systems exactly, one of the key goals of this package is
to provide a framework for other solvers to benchmark the reliability and
stability of their linear solvers, as our final solution vector $x$ is
guaranteed to be exact. In addition, SLIP LU provides a wrapper class for the
GNU Multiple Precision Arithmetic (GMP) \cite{granlund2015gnu} and GNU Multiple
Precision Floating Point Reliable (MPFR) \cite{fousse2007mpfr} libraries in
order to prevent memory leaks and improve the overall stability of these
external libraries. SLIP LU is written in ANSI C and is accompanied by a MATLAB
interface.

For all primary computational routines in Section \ref{s:primary}, the input
argument $A$ must be stored in a compressed sparse column (CSC) matrix with
entries in \verb|mpz_t| type (referred to as CSC \verb|mpz_t| matrix
henceforth), while $b$ must be stored as a dense \verb|mpz_t| matrix (i.e., a
dense matrix with entries in \verb|mpz_t| type).  However, the original data
type of entries in the input matrix $A$ and right hand side (RHS)
vectors $b$ can be any one of: \verb|double|, \verb|int64_t|, \verb|mpq_t|,
\verb|mpz_t|, or \verb|mpfr_t|, and their format(s) are allowed to be 
CSC, sparse triplet or dense. A discussion of how to use these
matrix formats and data types in the \verb|SLIP_matrix|, and to
perform conversions between matrix types and formats
in Section \ref{ss:populate_Ab}.

The matrices $L$ and $U$ are computed using integer-preserving
routines with the big integer (\verb|mpz_t|) data types from the GMP Library
\cite{granlund2015gnu}. The matrices $L$ and $U$ are computed one column at a
time, where each column is computed via the sparse REF triangular solve
detailed in \cite{lourenco2019exact}. All divisions performed in the algorithm
are guaranteed to be exact (i.e., integer); therefore, no greatest common
divisor algorithms are needed to reduce the size of entries.

The permutation matrices $P$ and $Q$ define the pivot ordering; $Q$ is the
fill-reducing column ordering, and $P$ is determined dynamically during the
factorization.  For the matrix $P$, the default option is to use a partial
pivoting scheme in which the diagonal entry in column $k$ is selected if it is
the same magnitude as the smallest entry of $k$-th column, otherwise the
smallest entry is selected as the $k$-th pivot. In addition to this approach,
the code allows diagonal pivoting, partial pivoting which selects the largest
pivot, or various tolerance based diagonal pivoting schemes. For the matrix
$Q$, the default ordering is the Column Approximate Minimum Degree (COLAMD)
algorithm \cite{davis2004algorithmcolamd,davis2004column}. Other approaches
include using the Approximate Minimum Degree (AMD) ordering
\cite{amestoy1996approximate,amestoy2004algorithmamd}, or no ordering ($Q=I$).
A discussion of how to select these permutations prior to factorization is
given in Section \ref{s:primary}.

Once the factorization $L D U = P A Q $ is computed, the solution vector 
$x$ is computed via sparse REF forward and backward substitution. 
The forward substitution is a variant of the sparse REF triangular solve 
discussed above. The backward substitution is a typical column oriented 
sparse backward substitution. Both of these routines require $b$ 
stored as a dense \verb|mpz_t| matrix. At the conclusion of the forward and 
backward substitution routines, the final solution vector(s) $x$ 
are guaranteed to be exact.  The solution $x$ is returned as a dense
\verb|mpq_t| matrix.

Using the SLIP matrix copy function, any matrix in any of the 15 combinations
of the set (CSC, triplet, dense) $\times$ (\verb|mpz_t|, \verb|mpq_t| ,
\verb|mpfr_t|, \verb|int64_t|, or \verb|double|), can be copied and converted
into any one of the 15 combinations.

One key advantage of utilizing SLIP LU with floating-point output is that the
solution is guaranteed to be exact until this final conversion; meaning that
roundoff errors are only introduced in the final conversion from rational
numbers. Thus, the solution $x$ output in \verb|double| precision are accurate
to machine roundoff (approximately $10^{-16}$) and SLIP LU utilizes higher
precision for the MPFR output; thus it is also accurate to user-specified
precision.

Most routines expect the input sparse matrix $A$ to be stored in CSC format.
This data structure stores the matrix $A$ as a sequence of three arrays:

\begin{itemize}
\item
\verb|A->p|: Column pointers; an array of size \verb|n+1|. The row indices of
column $j$ are located in positions \verb|A->p[j]| to \verb|A->p[j+1]-1| of the
array \verb|A->i|. Data type: \verb|int64_t|.

\item
\verb|A->i|: Row indices; an array of size equal to the number of entries in
the matrix. The entry \verb|A->i[k]| is the row index of the $k$th nonzero in
the matrix. Data type: \verb|int64_t|.

\item
\verb|A->x|: Numeric entries. The entry \verb|A->x[k]| is the numeric value of
the $k$th nonzero in the matrix.  The array \verb|A->x| has a union type, and
must be accessed via a suffix according to the type of \verb|A|.  For details,
see Section~\ref{ss:SLIP_matrix}.

\end{itemize}

An example matrix $A$ with \verb|mpz_t| type is stored as follows (notice that
via C convention, the indexing is zero based).
\[
A = \begin{bmatrix}
1 & 0 & 0 & 1 \\
2 & 0 & 4 & 12 \\
7 & 1 & 1 & 1 \\
0 & 2 & 3 & 0 \\
\end{bmatrix}
\]

\begin{verbatim}
A->p     = [0, 3, 5, 8, 11]
A->i     = [0, 1, 2, 2, 3, 1, 2, 3, 0,  1, 2]
A->x.mpz = [1, 2, 7, 1, 2, 4, 1, 3, 1, 12, 1]
\end{verbatim}

For example, the last column appears in positions 8 to 10 of \verb|A->i| and
\verb|A->x.mpz|, with row indices 0, 1, and 2, and values $a_{03}=1$,
$a_{13}=12$, and $a_{23}=1$.

%-------------------------------------------------------------------------------
\section{Availability}
%-------------------------------------------------------------------------------

\textbf{Copyright:} This software is copyright by Christopher Lourenco, Jinhao
Chen, Erick Moreno-Centeno, and Timothy Davis.

\noindent \textbf{Contact Info:} Contact Chris Lourenco,
\href{mailto:chrisjlourenco@gmail.com}{chrisjlourenco@gmail.com}, or Tim Davis,
\href{mailto:timdavis@aldenmath.com}{timdavis@aldenmath.com},
\href{mailto:davis@tamu.edu}{davis@tamu.edu}, or
\href{DrTimothyAldenDavis@gmail.com}{DrTimothyAldenDavis@gmail.com}

\noindent \textbf{License:} This software package is dual licensed under the
GNU General Public License version 2 or the GNU Lesser General Public License
version 3. Details of this license are in \verb|SLIP_LU/License/license.txt|.
For alternative licenses, please contact the authors.

\noindent \textbf{Location:} \url{https://github.com/clouren/SLIP_LU} and
\url{www.suitesparse.com}

\noindent \textbf{Required Packages:} SLIP LU requires the installation of AMD
\cite{amestoy1996approximate,amestoy2004algorithmamd}, COLAMD
\cite{davis2004column,davis2004algorithmcolamd}, \verb|SuiteSparse_config|
\cite{davis2020suitesparse}, the GNU GMP \cite{granlund2015gnu} and GNU MPFR
\cite{fousse2007mpfr} libraries.  AMD and COLAMD are available under a BSD
3-clause license, and no license restrictions apply to
\verb|SuiteSparse_config|.  Notice that AMD, COLAMD, and
\verb|SuiteSparse_config| are included in this distribution for
convenience. The GNU GMP and GNU MPFR library can be acquired and installed
from \url{https://gmplib.org/} and \url{http://www.mpfr.org/} respectively.

With a Debian/Ubuntu based Linux system, a compatible version of GMP and MPFR
can be installed with the following terminal commands:

{\small
\begin{verbatim}
    sudo apt-get install libgmp3-dev
    sudo apt-get install libmpfr-dev libmpfr-doc libmpfr4 libmpfr4-dbg
\end{verbatim} }

%-------------------------------------------------------------------------------
\section{Installation} \label{s:install}
%-------------------------------------------------------------------------------

Installation of SLIP LU requires the \verb|make| utility in Linux/MacOS, or
\verb|Cygwin make| in Windows. With the proper compiler, typing \verb|make|
under the main directory will compile AMD, COLAMD and SLIP LU to the respective
\verb|SLIP_LU/Lib| folder. To further install the libraries onto your computer,
simply type \verb|make install|.  Thereafter, to use the code inside of your
program, precede your code with \verb|#include "SLIP_LU.h"|.

To run the statement coverage tests, go to the \verb|Tcov| folder and
type \verb|make|.  The last line of output should read:

\begin{verbatim}
    statments not yet tested: 0
\end{verbatim}

If you want to use SLIP LU within MATLAB, from your installation of MATLAB,
\verb|cd| to the folder \verb|SLIP_LU/SLIP_LU/MATLAB| then type
\verb|SLIP_install|. This should compile the necessary code so that you can use
the \verb|SLIP_backslash| function from within MATLAB. Note that
\verb|SLIP_install| does not add the correct directory to your path; therefore,
if you want to use \verb|SLIP_backslash| in future sessions, type
\verb|pathtool| and save your path for future MATLAB sessions. If you cannot
save your path because of file permissions, edit your \verb|startup.m| by
adding \verb|addpath| commands (type \verb|doc startup| and \verb|doc addpath|
for more information).

%-------------------------------------------------------------------------------
\section{Managing the SLIP LU environment} \label{s:user:setup}
%-------------------------------------------------------------------------------

%-------------------------------------------------------------------------------
\cprotect\subsection{\verb|SLIP_LU_VERSION|: the software package version}
%-------------------------------------------------------------------------------

SLIP LU defines the following strings with \verb|#define|. Refer to
the \verb|SLIP_LU.h| file for details.

%----------------------------------------
\begin{center}
\begin{tabular}{ll}
\hline
Macro & purpose \\
\hline
\verb|SLIP_LU_VERSION|       & current version of the code (as a string)\\
\verb|SLIP_LU_VERSION_MAJOR| & major version of the code\\
\verb|SLIP_LU_VERSION_MINOR| & minor version of the code   \\
\verb|SLIP_LU_VERSION_SUB|   & sub version of the code\\
\hline
\end{tabular}
\end{center}

%-------------------------------------------------------------------------------
\cprotect\subsection{\verb|SLIP_info|: status code returned by SLIP LU}
\label{ss:SLIP_info}
%-------------------------------------------------------------------------------

Most SLIP LU functions return their status to the caller as their return value,
an enumerated type called \verb|SLIP_info|. All current possible values for
\verb|SLIP_info| are listed as follows:

\begin{center}
\begin{tabular}{rll}
\hline
    0& \verb|SLIP_OK|& The function was successfully executed.\\
\hline
    -1& \verb|SLIP_OUT_OF_MEMORY|& out of memory\\
\hline
    -2& \verb|SLIP_SINGULAR|& The input matrix $A$ is exactly singular.\\
\hline
    -3& \verb|SLIP_INCORRECT_INPUT|& One or more input arguments are incorrect.\\
\hline
    -4& \verb|SLIP_INCORRECT|& The solution is incorrect.\\
\hline
    -5& \verb|SLIP_PANIC| & SLIP LU environment error \\
\hline
\end{tabular}
\end{center}

Either \verb|SLIP_initialize| or \verb|SLIP_initialize_expert| (but not both)
must be called prior to using any other SLIP LU function.  \verb|SLIP_finalize|
must be called as the last SLIP LU function.

Subsequent SLIP LU sessions can be restarted after a call to
\verb|SLIP_finalize|, by calling either \verb|SLIP_initialize| or
\verb|SLIP_initialize_expert| (but not both), followed by a final call to
\verb|SLIP_finalize| when finished.

%-------------------------------------------------------------------------------
\cprotect\subsection{\verb|SLIP_initialize|: initialize the working environment}
%-------------------------------------------------------------------------------

\begin{mdframed}[userdefinedwidth=6in]
{\footnotesize
\begin{verbatim}
    SLIP_info SLIP_initialize
    (
        void
    ) ;
\end{verbatim}
} \end{mdframed}

\verb|SLIP_initialize| initializes the working environment for SLIP LU
functions.  SLIP LU utilizes a specialized memory management scheme in order to
prevent potential memory failures caused by GMP and MPFR libraries.  Either
this function or \verb|SLIP_initialize_expert| must be called prior to using
any other function in the library.  Returns \verb|SLIP_PANIC| if SLIP LU has
already been initialized, or \verb|SLIP_OK| if successful.

%-------------------------------------------------------------------------------
\cprotect\subsection{\verb|SLIP_initialize_expert|: initialize environment
(expert version)}\label{ss:SLIP_initialize_expert}
%-------------------------------------------------------------------------------

\begin{mdframed}[userdefinedwidth=6in]
{\footnotesize
\begin{verbatim}
    SLIP_info SLIP_initialize_expert
    (
        void* (*MyMalloc) (size_t),             // user-defined malloc
        void* (*MyCalloc) (size_t, size_t),     // user-defined calloc
        void* (*MyRealloc) (void *, size_t),    // user-defined realloc
        void  (*MyFree) (void *)                // user-defined free
    ) ;
\end{verbatim}
} \end{mdframed}

\verb|SLIP_initialize_expert| is the same as \verb|SLIP_initialize| except that
it allows for a redefinition of custom memory functions that are used for SLIP
LU and GMP/MPFR.  The four inputs to this function are pointers to four
functions with the same signatures as the ANSI C \verb'malloc', \verb'calloc',
\verb'realloc', and \verb'free' functions.  That is:

\begin{mdframed}[userdefinedwidth=6in]
{\footnotesize
\begin{verbatim}
    #include <stdlib.h>
    void *malloc (size_t size) ;
    void *calloc (size_t nmemb, size_t size) ;
    void *realloc (void *ptr, size_t size) ;
    void free (void *ptr) ;
\end{verbatim}
} \end{mdframed}

Returns \verb|SLIP_PANIC| if SLIP LU has already been initialized,
or \verb|SLIP_OK| if successful.

%-------------------------------------------------------------------------------
\cprotect\subsection{\verb|SLIP_finalize|: free the working environment}
\label{ss:SLIP_finalize}
%-------------------------------------------------------------------------------

\begin{mdframed}[userdefinedwidth=6in]
{\footnotesize
\begin{verbatim}
    SLIP_info SLIP_finalize
    (
        void
    ) ;
\end{verbatim}
} \end{mdframed}

\verb|SLIP_finalize| finalizes the working environment for SLIP LU
library, and frees any internal workspace created by SLIP LU.  It must be
called as the last \verb|SLIP_*| function called, except that a subsequent
call to \verb|SLIP_initialize*| may be used to start another SLIP LU session.
Returns \verb|SLIP_PANIC| if SLIP LU has not been initialized,
or \verb|SLIP_OK| if successful.

%-------------------------------------------------------------------------------
\section{Memory Management} \label{s:user:memmanag}
%-------------------------------------------------------------------------------

The routines in this section are used to allocate and free memory for the data
structures used in SLIP LU.  By default, SLIP LU relies on the SuiteSparse
memory management functions, \verb|SuiteSparse_malloc|,
\verb|SuiteSparse_calloc|, \verb|SuiteSparse_realloc|, and
\verb|SuiteSparse_free|.  By default, those functions rely on the ANSI C
\verb|malloc|, \verb|calloc|, \verb|realloc|, and \verb|free|, but this may be
changed by initializing the SLIP LU environment with
\verb|SLIP_initialize_expert|.

%-------------------------------------------------------------------------------
\newpage
\cprotect\subsection{\verb|SLIP_calloc|: allocate initialized memory}
\label{ss:SLIP_calloc}
%-------------------------------------------------------------------------------

\begin{mdframed}[userdefinedwidth=6in]
{\footnotesize
\begin{verbatim}
    void *SLIP_calloc 
    ( 
        size_t nitems,      // number of items to allocate 
        size_t size         // size of each item 
    ) ;
\end{verbatim}
} \end{mdframed}

\verb|SLIP_calloc| allocates a block of memory for an array of \verb|nitems|
elements, each of them \verb|size| bytes long, and initializes all its bits to
zero. If any input is less than 1, it is treated as if equal to 1. If the
function failed to allocate the requested block of memory, then a \verb|NULL|
pointer is returned.
Returns \verb|NULL| if SLIP LU has not been initialized.

%-------------------------------------------------------------------------------
\cprotect\subsection{\verb|SLIP_malloc|: allocate uninitialized memory}
\label{ss:SLIP_malloc}
%-------------------------------------------------------------------------------

\begin{mdframed}[userdefinedwidth=6in]
{\footnotesize
\begin{verbatim}
    void *SLIP_malloc
    (
        size_t size        // size of memory space to allocate
    ) ;
\end{verbatim}
} \end{mdframed}

\verb|SLIP_malloc| allocates a block of \verb|size| bytes of memory, returning
a pointer to the beginning of the block. The content of the newly allocated
block of memory is not initialized, remaining with indeterminate values.
If \verb|size| is less than 1, it is treated as if equal to 1. If the function
fails to allocate the requested block of memory, then a \verb|NULL| pointer is
returned.
Returns \verb|NULL| if SLIP LU has not been initialized.

%-------------------------------------------------------------------------------
\cprotect\subsection{\verb|SLIP_realloc|: resize allocated memory}
\label{ss:SLIP_realloc}
%-------------------------------------------------------------------------------

\begin{mdframed}[userdefinedwidth=6in]
{\footnotesize
\begin{verbatim}
    void *SLIP_realloc      // pointer to reallocated block, or original block
                            // if the realloc failed
    (
        int64_t nitems_new,     // new number of items in the object
        int64_t nitems_old,     // old number of items in the object
        size_t size_of_item,    // sizeof each item
        void *p,                // old object to reallocate
        bool *ok                // true if success, false on failure
    ) ;
\end{verbatim}
} \end{mdframed}

\verb|SLIP_realloc| is a wrapper for realloc.  If \verb|p| is non-\verb|NULL| on
input, it points to a previously allocated object of size \verb|nitems_old|
$\times$ \verb|size_of_item|.  The object is reallocated to be of size
\verb|nitems_new| $\times$ \verb|size_of_item|.  If \verb|p| is \verb|NULL| on input,
then a new object of that size is allocated.  On success, a pointer to the new
object is returned.  Returns \verb|ok| as \verb|false| if SLIP LU has not been
initialized.

If the reallocation fails, \verb|p| is not modified, and \verb|ok| is returned
as \verb|false| to indicate that the reallocation failed.  If the size
decreases or remains the same, then the method always succeeds (\verb|ok| is
returned as \verb|true|), unless SLIP LU has not been initialized.

Typical usage:  the following code fragment allocates an array of 10
\verb|int|'s, and then increases the size of the array to 20 \verb|int|'s.  If
the \verb|SLIP_malloc| succeeds but the \verb|SLIP_realloc| fails, then the
array remains unmodified, of size 10.

\begin{mdframed}[userdefinedwidth=6in]
{\footnotesize
\begin{verbatim}
     int *p ;
     p = SLIP_malloc (10 * sizeof (int)) ;
     if (p == NULL) { error here ... }
     printf ("p points to an array of size 10 * sizeof (int)\n") ;
     bool ok ;
     p = SLIP_realloc (20, 10, sizeof (int), p, &ok) ;
     if (ok) printf ("p has size 20 * sizeof (int)\n") ;
     else printf ("realloc failed; p still has size 10 * sizeof (int)\n") ;
     SLIP_free (p) ;
\end{verbatim}
} \end{mdframed}

%-------------------------------------------------------------------------------
\cprotect\subsection{\verb|SLIP_free|: free allocated memory}
\label{ss:SLIP_free}
%-------------------------------------------------------------------------------

\begin{mdframed}[userdefinedwidth=6in]
{\footnotesize
\begin{verbatim}
    void SLIP_free
    (
        void *p         // Pointer to memory space to free
    ) ;
\end{verbatim}
} \end{mdframed}

\verb|SLIP_free| deallocates the memory previously allocated by a call to
\verb|SLIP_calloc|, \verb|SLIP_malloc|, or \verb|SLIP_realloc|.  If \verb|p| is
\verb|NULL| on input, then no action is taken (this is not an error condition).
To guard against freeing the same memory space twice, the following macro
\verb|SLIP_FREE| is provided, which calls \verb|SLIP_free| and then sets the
freed pointer to \verb|NULL|.

\begin{mdframed}[userdefinedwidth=6in]
{\footnotesize
\begin{verbatim}
    #define SLIP_FREE(p)                        \
    {                                           \
        SLIP_free (p) ;                         \
        (p) = NULL ;                            \
    }
\end{verbatim}
} \end{mdframed}

No action is taken if SLIP LU has not been initialized.

%-------------------------------------------------------------------------------
\cprotect\section{The \verb|SLIP_options| object:
parameter settings for SLIP LU} \label{ss:SLIP_options}
%-------------------------------------------------------------------------------

The \verb|SLIP_options| object contains numerous parameters that may be
modified to change the behavior of the SLIP LU functions.  Default values of
these parameters will lead to good performance in most cases.  Modifying this
struct provides control of column orderings, pivoting schemes, and other
components of the factorization.

%-------------------------------------------------------------------------------
\cprotect\subsection{\verb|SLIP_pivot|: enum for pivoting schemes}
\label{ss:SLIP_pivot}
%-------------------------------------------------------------------------------

There are six available pivoting schemes provided in SLIP LU that can be
selected with the \verb|SLIP_options| structure.  If the matrix is non-singular
(in an exact sense), then the pivot is always nonzero, and is chosen as the
{\em smallest} nonzero entry, with the smallest magnitude.  This may seem
counter-intuitive, but selecting a small nonzero pivot leads to smaller growth
in the number of digits in the entries of \verb|L| and \verb|U|.  This choice
does not lead to any kind of numerical inaccuracy, since SLIP LU is guaranteed
to find an exact roundoff-error free factorization of a non-singular matrix
(unless it runs out of memory), for any nonzero pivot choice.

The pivot tolerance for two of the pivoting schemes is specified by the
\verb|tol| component in \verb|SLIP_options|.  The pivoting schemes are as
follows:

%----------------------------------------
{\small
\begin{center}
\begin{tabular}{llp{4in}}
\hline
0 & \verb|SLIP_SMALLEST|        & The $k$-th pivot is selected as the smallest
                                  entry in the $k$-th column.\\
\hline
1 & \verb|SLIP_DIAGONAL|        & The $k$-th pivot is selected as the diagonal
                                  entry. If the diagonal entry is zero,
                                  this method instead selects the smallest
                                  pivot in the column.\\
\hline
2 & \verb|SLIP_FIRST_NONZERO|   & The $k$-th pivot is selected as the first
                                  eligible nonzero in the column. \\
\hline
3 & \verb|SLIP_TOL_SMALLEST|    & The $k$-th pivot is selected as the diagonal
                                  entry if the diagonal is within a
                                  specified tolerance of the smallest entry in
                                  the column. Otherwise, the smallest
                                  entry in the $k$-th column is selected.
                                  This is the default pivot selection
                                  strategy. \\
\hline
4 & \verb|SLIP_TOL_LARGEST|     & The $k$-th pivot is selected as the diagonal
                                  entry if the diagonal is within a
                                  specified tolerance of the largest entry in
                                  the column.  Otherwise, the largest
                                  entry in the $k$-th column is selected. \\
\hline
5 & \verb|SLIP_LARGEST|         & The $k$-th pivot is selected as the largest
                                  entry in the $k$-th column. \\
\hline
\end{tabular}
\end{center}
}

%-------------------------------------------------------------------------------
\cprotect\subsection{\verb|SLIP_col_order|: enum for column ordering schemes}
\label{ss:SLIP_col_order}
%-------------------------------------------------------------------------------

The SLIP LU library provides three column ordering schemes: no pre-ordering,
COLAMD, and AMD, selected via the \verb|order|
component in the \verb|SLIP_options| structure described in Section
\ref{ss:SLIP_options_struct}.

{\small
\begin{center}
\begin{tabular}{llp{4in}}
\hline
0 & \verb|SLIP_NO_ORDERING| & No pre-ordering is performed on the matrix $A$,
                              that is $Q = I$. \\
\hline
1 & \verb|SLIP_COLAMD|      & The columns of $A$ are permuted prior to
                              factorization using the COLAMD
                              \cite{davis2004algorithmcolamd} ordering.
                              This is the default ordering. \\
\hline
2 & \verb|SLIP_AMD|         & The nonzero pattern of $A + A^T$ is analyzed and
                              the columns of $A$ are permuted prior to
                              factorization based on the AMD
                              \cite{amestoy2004algorithmamd} ordering of
                              $A+A^T$. This works well if $A$ has a mostly
                              symmetric pattern, but tends to be worse
                              than COLAMD on matrices with unsymmetric pattern.
                              \cite{davis2004column}.\\
\hline
\end{tabular}
\label{tab:SLIP_pivot}
\end{center}
}

%-------------------------------------------------------------------------------
\cprotect\subsection{ \verb|SLIP_options| structure}
\label{ss:SLIP_options_struct}
%-------------------------------------------------------------------------------

The \verb|SLIP_options| struct stores key command parameters for various
functions used in the SLIP LU package. The \verb|SLIP_options* option| struct
contains the following components:

\begin{itemize}
\item
\verb|option->pivot|: An enum \verb|SLIP_pivot| type (discussed in Section
\ref{ss:SLIP_pivot}) which controls the type of pivoting used. Default value:
\verb|SLIP_TOL_SMALLEST| (3).

\item
\verb|option->order|: An enum \verb|SLIP_col_order| type (discussed in Section
\ref{ss:SLIP_col_order}) which controls what column ordering is used. Default
value: \verb|SLIP_COLAMD| (1).

\item
\verb|option->tol|: A \verb|double| tolerance for
the tolerance-based pivoting scheme, i.e., \verb|SLIP_TOL_SMALLEST| or
\verb|SLIP_TOL_LARGEST|. \verb|option->tol| must be in the range of $(0,1]$.
Default value: 1 meaning that the diagonal entry will be selected if it has the
same magnitude as the smallest entry in the $k$ the column.

\item
\verb|option->print_level|: An \verb|int| which controls the amount of
output:
0: print nothing, 1: just errors, 2: terse, with basic stats from
COLAMD/AMD and SLIP, 3: all, with matrices and results. Default value: 0.

\item
\verb|option->prec|: An \verb|int32_t| which specifies the precision used
for multiple precision floating point numbers, (i.e., MPFR). This
can be any integer larger than \verb|MPFR_PREC_MIN| (value of 1 in MPFR 4.0.2
and 2 in some legacy versions) and smaller than \verb|MPFR_PREC_MAX| (usually
the largest possible integer available in your system). Default value: 128
(quad precision).

\item
\verb|option->round|: A \verb|mpfr_rnd_t| which determines the type
of MPFR rounding to be used by SLIP LU. This is a parameter of the MPFR
library. The options for this parameter are:

    \begin{itemize}
        \item \verb|MPFR_RNDN|: round to nearest
            (roundTiesToEven in IEEE 754-2008)
        \item \verb|MPFR_RNDZ|: round toward zero
            (roundTowardZero in IEEE 754-2008)
        \item \verb|MPFR_RNDU|: round toward plus infinity
            (roundTowardPositive in IEEE 754-2008)
        \item \verb|MPFR_RNDD|: round toward minus infinity
            (roundTowardNegative in IEEE 754-2008)
        \item \verb|MPFR_RNDA|: round away from zero
        \item \verb|MPFR_RNDF|: faithful rounding. This is not stable.
    \end{itemize}

\noindent Refer to the MPFR User Guide available at
\url{https://www.mpfr.org/mpfr-current/mpfr.pdf} for details on the MPFR
rounding style and any other utilized MPFR convention. Default value:
\verb|MPFR_RNDN|.

\item
\verb|option->check|: A \verb|bool| which indicates whether the solution to the
system should be checked. Intended for debugging only; the SLIP LU library is
guaranteed to return the exact solution. Default value: \verb|false|.

\end{itemize}

All SLIP LU routines except basic memory management routines in Sections
\ref{ss:SLIP_finalize}-\ref{ss:SLIP_calloc} and \verb|SLIP_options| allocation
routine in \ref{ss:create_default_options} require \verb|option| as an input
argument.  The construction of the \verb|option| struct can be avoided by
passing \verb|NULL| for the default settings.  Otherwise, the following
functions create and destroy a \verb|SLIP_options| object:

%----------------------------------------
\begin{center}
\begin{tabular}{lp{2.5in}l}
\hline
function/macro name & description & section \\
\hline
\verb|SLIP_create_default_options|
    & create and return \verb|SLIP_options| pointer
      with default parameters upon successful allocation
    & \ref{ss:create_default_options} \\
\hline
\verb|SLIP_FREE|
    & destroy \verb|SLIP_options| object
    & \ref{ss:SLIP_free} \\
\hline
\end{tabular}
\end{center}

%-------------------------------------------------------------------------------
\cprotect\subsection{\verb|SLIP_create_default_options|: create default \verb|SLIP_options| object}
\label{ss:create_default_options}
%-------------------------------------------------------------------------------

\begin{mdframed}[userdefinedwidth=6in]
{\footnotesize
\begin{verbatim}
    SLIP_options* SLIP_create_default_options
    (
        void
    ) ;
\end{verbatim}
} \end{mdframed}

\verb|SLIP_create_default_options| creates and returns a pointer to a
\verb|SLIP_options| struct with default parameters upon successful allocation,
which are discussed in Section \ref{ss:SLIP_options_struct}.  To safely free
the \verb|SLIP_options* option| structure, simply use \newline \verb|SLIP_FREE(option)|.
All functions that require \verb|SLIP_options *option| as an input argument can
have a \verb'NULL' pointer passed instead. In this case, the default value of
the corresponding command option is used. For example, if a \verb'NULL' pointer
is passed to the symbolic analysis routines, COLAMD is used. As a result,
defaults are desired, the \verb|SLIP_options| struct need not be allocated.
Returns \verb|NULL| if SLIP LU has not been initialized.

%-------------------------------------------------------------------------------
\cprotect\section{The \verb|SLIP_matrix| object} \label{ss:SLIP_matrix}
%-------------------------------------------------------------------------------

All matrices for SLIP LU are stored as a \verb|SLIP_matrix| object (a pointer
to a \verb'struct').  The matrix can be held in three formats: CSC, triplet or
dense matrix (as discussed in Section \ref{ss:SLIP_kind}) with entries stored
as 5 different types: \verb|mpz_t|, \verb|mpq_t|, \verb|mpfr_t|, \verb|int64_t|
and \verb|double| (as discussed in Section \ref{ss:SLIP_type}).  This gives a
total of 15 different combinations of matrix format and entry type. Note that
not all functions accept all 15 matrix types. Indeed, most functions expect the
input matrix $A$ to be a CSC \verb|mpz_t| matrix while vectors (such as $x$ and
$b$) are in dense format.

%-------------------------------------------------------------------------------
\cprotect\subsection{\verb|SLIP_kind|: enum for matrix formats}
\label{ss:SLIP_kind}
%-------------------------------------------------------------------------------

The SLIP LU library provides three available matrix formats: sparse CSC
(compressed sparse column), sparse triplet and dense.

{\small
\begin{center}
\begin{tabular}{llp{4in}}
\hline
0 & \verb|SLIP_CSC| & Matrix is in compressed sparse column format. \\
\hline
1 & \verb|SLIP_TRIPLET|      & Matrix is in sparse triplet format. \\
\hline
2 & \verb|SLIP_DENSE|        & Matrix is in dense format.\\
\hline
\end{tabular}
\label{tab:SLIP_kind}
\end{center}
}

%-------------------------------------------------------------------------------
\cprotect\subsection{\verb|SLIP_type|: enum for data types of matrix entry}
\label{ss:SLIP_type}
%-------------------------------------------------------------------------------

The SLIP LU library provides five data types for matrix entries: \verb|mpz_t|,
\verb|mpq_t|, \verb|mpfr_t|, \verb|int64_t| and \verb|double|.

{\small
\begin{center}
\begin{tabular}{llp{4in}}
\hline
0 & \verb|SLIP_MPZ|     & Matrix entries are in \verb|mpz_t| type: an integer
                          of arbitrary size. \\
\hline
1 & \verb|SLIP_MPQ|     & Matrix entries are in \verb|mpq_t| type: a rational
                          number with arbitrary-sized integer numerator and
                          denominator. \\
\hline
2 & \verb|SLIP_MPFR|    & Matrix entries are in \verb|mpfr_t| type: a
                          floating-point number of arbitrary precision. \\
\hline
3 & \verb|SLIP_INT64|   & Matrix entries are in \verb|int64_t| type. \\
\hline
4 & \verb|SLIP_FP64|    & Matrix entries are in \verb|double| type. \\
\hline
\end{tabular}
\label{tab:SLIP_type}
\end{center}
}

%-------------------------------------------------------------------------------
\cprotect\subsection{\verb|SLIP_matrix| structure}
%-------------------------------------------------------------------------------

A matrix \verb|SLIP_matrix *A| has the following components:

\begin{itemize}
\item \verb|A->m|: Number of rows in the matrix. Data Type: \verb|int64_t|.

\item \verb|A->n|: Number of columns in the matrix. Data Type: \verb|int64_t|.

\item \verb|A->nz|: The number of nonzeros in the matrix $A$, if $A$ is
a triplet matrix (ignored for matrices in CSC or dense formats). Data Type:
\verb|int64_t|.

\item \verb|A->nzmax|: The allocated size of the vectors \verb|A->i|,
\verb|A->j| and \verb|A->x|. Note that \verb|A->nzmax| $\geq$ \verb|nnz(A)|,
where \verb|nnz(A)| is the return value of \verb|SLIP_matrix_nnz(A,option)|.
Data Type: \verb|int64_t|.

\item \verb|A->kind|: Indicating the kind of matrix A: CSC, triplet or dense.
Data Type: \verb|SLIP_kind|.

\item \verb|A->type|: Indicating the type of entries in matrix A: \verb|mpz_t|,
\verb|mpq_t|, \verb|mpfr_t|, \verb|int64_t| or \verb|double|.
Data Type: \verb|SLIP_type|.

\item \verb|A->p|: An array of size \verb|A->n|$+1$ which contains column pointers
of $A$, if $A$ is a CSC matrix (\verb|NULL| for matrices in triplet or dense
formats). Data Type: \verb|int64_t*|.

\item \verb|A->p_shallow|: A boolean indicating whether \verb|A->p| is shallow.
Data Type: \verb|bool|.

\item \verb|A->i|: An array of size \verb|A->nzmax| which contains the row
indices of the nonzeros in $A$, if $A$ is a CSC or triplet matrix (\verb|NULL|
for dense matrices). The matrix is zero-based, so row indices are
in the range of $[0,$ \verb|A->m|$-1]$. Data Type: \verb|int64_t*|.

\item \verb|A->i_shallow|: A boolean indicating whether \verb|A->i| is shallow.
Data Type: \verb|bool|.

\item \verb|A->j|: An array of size \verb|A->nzmax| which contains the column
indices of the nonzeros in $A$, if $A$ is a triplet matrix (\verb|NULL| for
matrices in CSC or dense formats).
The matrix is zero-based, so column indices are
in the range of $[0,$ \verb|A->n|$-1]$. Data Type: \verb|int64_t*|.

\item \verb|A->j_shallow|: A boolean indicating whether \verb|A->j| is shallow.
Data Type: \verb|bool|.

\item \verb|A->x|: An array of size \verb|A->nzmax| which contains the
numeric values of the matrix.  This array is a union, and must be accessed via
one of: \verb|A->x.mpz|, \verb|A->x.mpq|, \verb|A->x.mpfr|, \verb|A->x.int64|,
or \verb|A->x.fp64|, depending on the \verb|A->type| parameter.
Data Type: \verb|union|.

\item \verb|A->x_shallow|: A boolean indicating whether \verb|A->x| is
shallow. Data Type: \verb|bool|.

\item \verb|A->scale|: A scaling parameter for matrix of \verb|mpz_t| type. For
all matrices whose entries are stored in data type other than \verb|mpz_t|,
\verb|A->scale = 1|. This is used to ensure that entry can be represented as an
integer in an \verb|mpz_t| matrix if these entries are converted from non-integer type
data (such as double, variable precision floating point, or rational). Data
Type: \verb|mpq_t|.

\end{itemize}

Specifically, for different kinds of \verb|A| of size \verb|A->m| $\times$ \verb|A->n|
with \verb|nz| nonzero entries, its components are defined as:

\begin{itemize}
\item
 (0) \verb|SLIP_CSC|:  A sparse matrix in CSC (compressed sparse column) format.
      \verb|A->p| is an \verb|int64_t| array of size \verb|A->n|+1, \verb|A->i|
      is an \verb|int64_t| array of size \verb|A->nzmax| (with $nz$ $\le$
      \verb|A->nzmax|), and \verb|A->x.TYPE| is an array of size
      \verb|A->nzmax| of matrix entries (\verb'TYPE' is one of \verb|mpz|,
      \verb|mpq|, \verb|mpfr|, \verb|int64|, or \verb|fp64|).  The row indices
      of column $j$ appear in \verb|A->i [A->p [j] ... A->p [j+1]-1]|, and the
      values appear in the same locations in \verb|A->x.TYPE|.  The \verb|A->j|
      array is \verb|NULL|.  \verb|A->nz| is ignored; the number of entries in
      \verb|A| is given by \verb|A->p [A->n]|.
      Row indices need not be sorted in each column, but duplicates cannot
      appear.

\item
 (1) \verb|SLIP_TRIPLET|:  A sparse matrix in triplet format.  \verb|A->i| and
     \verb|A->j| are both \verb|int64_t| arrays of size \verb|A->nzmax|, and
     \verb|A->x.TYPE| is an array of values of the same size.  The $k$th tuple
     has row index \verb|A->i [k]|, column index \verb|A->j [k]| , and value
     \verb|A->x.TYPE [k]|, with 0 $\le$ $k <$ \verb|A->nz|.
     The \verb|A->p| array is \verb|NULL|.
     Triplets can be unsorted, but duplicates cannot appear.

\item
 (2) \verb|SLIP_DENSE|:  A dense matrix.  The integer arrays \verb|A->p|,
     \verb|A->i|, and \verb|A->j| are all \verb|NULL|.  \verb|A->x.TYPE| is a
     pointer to an array of size \verb|A->m|*\verb|A->n|, stored in
     column-oriented format.  The value of $A(i,j)$ is \verb|A->x.TYPE [p]|
     with \verb|p| = $i + j*$\verb|A->m|.  \verb|A->nz| is ignored; the number
     of entries in \verb|A| is \verb|A->m| $\times$ \verb|A->n|.

\end{itemize}

\verb|A| may contain shallow components, \verb|A->p|, \verb|A->i|, \verb|A->j|,
and \verb|A->x|.  For example, if \verb|A->p_shallow| is true, then a
non-\verb|NULL| \verb|A->p| is a pointer to a read-only array, and the
\verb|A->p| array is not freed by \verb|SLIP_matrix_free|.  If \verb|A->p| is
\verb|NULL| (for a triplet or dense matrix), then \verb|A->p_shallow| has no
effect.

To simplify the access the entries in \verb|A|, SLIP LU package provides the
following macros (Note that the \verb|TYPE| parameter in the macros is one of:
\verb|mpz|, \verb|mpq|, \verb|mpfr|, \verb|int64| or \verb|fp64|):

\begin{itemize}

\item
\verb|SLIP_1D(A,k,TYPE)|: used to access the $k$th entry in
                         \verb|SLIP_matrix *A| using 1D linear addressing for
                         any matrix kind (CSC, triplet or dense), in any type
                         with \verb|TYPE| specified corresponding

\item
\verb|SLIP_2D(A,i,j,TYPE)|: used to access the $(i,j)$th entry in a dense
                            \verb|SLIP_matrix *A|.

\end{itemize}

The SLIP LU package has a set of functions to allocate, copy(convert), query and
destroy a SLIP LU matrix, \verb|SLIP_matrix|, as shown in the following table.

%----------------------------------------
{\small
\begin{center}
\begin{tabular}{lp{2.5in}l}
\hline
function name & description & section \\
\hline
\verb|SLIP_matrix_allocate|
    & allocate a $m$-by-$n$ \verb|SLIP_matrix|
    & \ref{s:user:matrix_allocate} \\
\hline
\verb|SLIP_matrix_free|
    & destroy a \verb|SLIP_matrix| and free its allocated memory
    & \ref{s:user:matrix_free} \\
\hline
\verb|SLIP_matrix_copy|
    & make a copy of a matrix, into another kind and/or type
    & \ref{s:user:matrix_copy} \\
\hline
\verb|SLIP_matrix_nnz|
    & get the number of entries in a matrix
    & \ref{s:user:matrix_nnz} \\
\hline
\end{tabular}
\end{center}
}

%-------------------------------------------------------------------------------
\newpage
\cprotect\subsection{\verb|SLIP_matrix_allocate|: allocate a $m$-by-$n$
\verb|SLIP_matrix|}
\label{s:user:matrix_allocate}
%-------------------------------------------------------------------------------

\begin{mdframed}[userdefinedwidth=6in]
{\footnotesize
\begin{verbatim}
    SLIP_info SLIP_matrix_allocate
    (
        SLIP_matrix **A_handle, // matrix to allocate
        SLIP_kind kind,         // CSC, triplet, or dense
        SLIP_type type,         // mpz, mpq, mpfr, int64, or double (fp64)
        int64_t m,              // # of rows
        int64_t n,              // # of columns
        int64_t nzmax,          // max # of entries
        bool shallow,           // if true, matrix is shallow.  A->p, A->i,
                                // A->j, A->x are all returned as NULL and must
                                // be set by the caller.  All A->*_shallow are
                                // returned as true.
        bool init,              // If true, and the data types are mpz, mpq, or
                                // mpfr, the entries are initialized (using the
                                // appropriate SLIP_mp*_init function). If
                                // false, the mpz, mpq, and mpfr arrays are
                                // allocated but not initialized.
        const SLIP_options *option
    ) ;
\end{verbatim}
} \end{mdframed}

\verb|SLIP_matrix_allocate| allocates memory space for a $m$-by-$n$
\verb|SLIP_matrix| whose kind (CSC, triplet or dense) and data type
(\verb|mpz|, \verb|mpq|, \verb|mpfr|, \verb|int64| or \verb|fp64|) is
specified. If \verb|shallow| is true, all components (\verb|A->p|, \verb|A->i|,
\verb|A->j|, \verb|A->x|) are returned as \verb|NULL|, and their shallow flags
are all true.  The pointers \verb|A->p|, \verb|A->i|, \verb|A->j|,
and/or \verb|A->x| can then be assigned from arrays in the calling application.

If \verb|shallow| is false, the appropriate individual arrays are allocated
(via \verb|SLIP_calloc|). The second boolean parameter is used if the entries
are \verb|mpz_t|, \verb|mpq_t|, or \verb|mpfr_t|. Specifically, if \verb|init|
is true, the individual entries within \verb|A->x.TYPE| are initialized using
the appropriate \verb|SLIP_mp*_init|) function. Otherwise, if \verb|init| is
false, the \verb|A->x.TYPE| array is allocated (via \verb|SLIP_calloc|) and
left that way.  They are not otherwise initialized, and attempting to access
the values of these uninitialized entries will lead to undefined behavior.
Returns \verb|SLIP_PANIC| if SLIP LU has not been initialized.

%-------------------------------------------------------------------------------
\cprotect\subsection{\verb|SLIP_matrix_free|: free a \verb|SLIP_matrix|}
\label{s:user:matrix_free}
%-------------------------------------------------------------------------------

\begin{mdframed}[userdefinedwidth=6in]
{\footnotesize
\begin{verbatim}
    SLIP_info SLIP_matrix_free
    (
        SLIP_matrix **A_handle, // matrix to free
        const SLIP_options *option
    ) ;
\end{verbatim}
} \end{mdframed}

\verb|SLIP_matrix_free| frees the \verb|SLIP_matrix *A|.  Note that the input
of the function is the pointer to the pointer of a \verb|SLIP_matrix|
structure. This is because this function internally sets the pointer of a
\verb|SLIP_matrix| to be \verb|NULL| to prevent potential segmentation fault
that could be caused by double \verb|free|.  If default settings are desired,
\verb|option| can be input as \verb|NULL|.

%-------------------------------------------------------------------------------
\cprotect\subsection{\verb|SLIP_matrix_copy|: make a copy of a \verb|SLIP_matrix|}
\label{s:user:matrix_copy}
%-------------------------------------------------------------------------------

\begin{mdframed}[userdefinedwidth=6in]
{\footnotesize
\begin{verbatim}
    SLIP_info SLIP_matrix_copy
    (
        SLIP_matrix **C,        // matrix to create (never shallow)
        // inputs, not modified:
        SLIP_kind kind,         // CSC, triplet, or dense
        SLIP_type type,         // mpz_t, mpq_t, mpfr_t, int64_t, or fp64
        SLIP_matrix *A,         // matrix to make a copy of (may be shallow)
        const SLIP_options *option
    ) ;
\end{verbatim}
} \end{mdframed}

\verb|SLIP_matrix_copy| creates a \verb|SLIP_matrix *C| which is a modified
copy of a \verb|SLIP_matrix *A|. This function can convert between any pair of
the 15 kinds of matrices, so the new matrix \verb|C| can be of any type or kind
different than \verb|A|.  On input \verb|C| must be non-\verb|NULL|, and the
value of \verb|*C| is ignored; it is overwritten, output with the matrix
\verb|C|, which is a copy of \verb|A| of kind \verb|kind| and type \verb|type|.

The input matrix is assumed to be valid. It can be checked first with
\verb|SLIP_matrix_check| (Section \ref{s:user:matrix_check}), if desired.
Results are undefined for an invalid input matrix \verb|A|.  Returns
\verb|SLIP_PANIC| if SLIP LU has not been initialized.

%-------------------------------------------------------------------------------
\cprotect\subsection{\verb|SLIP_matrix_nnz|: get the number of entries in a
\verb|SLIP_matrix|}
\label{s:user:matrix_nnz}
%-------------------------------------------------------------------------------

\begin{mdframed}[userdefinedwidth=6in]
{\footnotesize
\begin{verbatim}
    int64_t SLIP_matrix_nnz     // return # of entries in A, or -1 on error
    (
        const SLIP_matrix *A,         // matrix to query
        const SLIP_options *option
    ) ;
\end{verbatim}
} \end{mdframed}

\verb|SLIP_matrix_nnz| returns the number of entries in a \verb|SLIP_matrix *A|.
For details regarding how the number of entries is obtained for different kinds
of matrices, refer to Section \ref{ss:SLIP_matrix}.
For any matrix with invalid dimension(s), this function returns -1.
If default settings are desired, \verb|option| can be input as \verb|NULL|.
Returns -1 if SLIP LU has not been initialized.

%-------------------------------------------------------------------------------
\newpage
\cprotect\subsection{\verb|SLIP_matrix_check|: check and print a \verb|SLIP_matrix|}
\label{s:user:matrix_check}
%-------------------------------------------------------------------------------

\begin{mdframed}[userdefinedwidth=6in]
{\footnotesize
\begin{verbatim}
    SLIP_info SLIP_matrix_check     // returns a SLIP_LU status code
    (
        const SLIP_matrix *A,       // matrix to check
        const SLIP_options* option  // defines the print level
    ) ;
\end{verbatim}
} \end{mdframed}

\verb|SLIP_matrix_check| checks the validity of a \verb|SLIP_matrix *A| in any
of the 15 different matrix types (CSC, triplet, dense) $\times$ (\verb|mpz|,
\verb|mpq|, \verb|mpfr|, \verb|int64|, \verb|fp64|). The print level can be
changed via \verb|option->print_level| (refer to Section \ref{ss:SLIP_options}
for more details).  If default settings are desired, \verb|option| can be input
as \verb|NULL|.  Returns \verb|SLIP_PANIC| if SLIP LU has not been initialized.

%-------------------------------------------------------------------------------
\section{Primary Computational Routines}
\label{s:primary}
%-------------------------------------------------------------------------------

These routines perform symbolic analysis, compute the LU factorization of the
matrix $A$, and solve $Ax=b$ using the factorization of $A$.

%-------------------------------------------------------------------------------
\cprotect\subsection{\verb|SLIP_LU_analysis| structure}
\label{ss:SLIP_LU_analysis}
%-------------------------------------------------------------------------------

The \verb|SLIP_LU_analysis| data structure is used for storing the column
permutation for LU and the estimate of the number of nonzeros that may appear
in $L$ and $U$.  This need not be modified or accessed in the user application;
it simply needs to be passed in directly to the other functions that take it as
in input parameter. A \verb|SLIP_LU_analysis| structure has the following
components:

\begin{itemize}
\item \verb|S->q|: The column permutation stored as a dense \verb|int64_t|
vector of size $n+1$, where $n$ is the number of columns of the analyzed matrix.
Currently this vector is obtained via COLAMD, AMD, or is set to no ordering
(i.e., $[0, 1, \hdots, n-1]$).

\item \verb|S->lnz|: An \verb|int64_t| which is an estimate of the number of
nonzeros in $L$. \verb|S->lnz| must be in the range of $[n, n^2]$. If
\verb|S->lnz| is too small, the program may waste time performing extra memory
reallocations. This is set during the symbolic analysis.

\item \verb|S->unz|: An \verb|int64_t| which is an estimate of the number of
nonzeros in $U$. \verb|S->unz| must be in the range of $[n, n^2]$. If
\verb|S->unz| is too small, the program may waste time performing extra memory
reallocations. This is set during the symbolic analysis.
\end{itemize}

The SLIP LU package provides the following functions to create and destroy a
\verb|SLIP_LU_analysis| object:

%----------------------------------------
{\small
\begin{center}
\begin{tabular}{lll}
\hline
function/macro name & description & section \\
\hline
\verb|SLIP_LU_analyze|
    & create \verb|SLIP_LU_analysis| object
    & \ref{s:SLIP_LU_analyze} \\
\hline
\verb|SLIP_LU_analysis_free|
    & destroy \verb|SLIP_LU_analysis| object
    & \ref{ss:LU_analysis_free} \\
\hline
\end{tabular}
\end{center}
}

%-------------------------------------------------------------------------------
\cprotect\subsection{\verb|SLIP_LU_analyze|: perform symbolic analysis}
\label{s:SLIP_LU_analyze}
%-------------------------------------------------------------------------------

\begin{mdframed}[userdefinedwidth=6in]
{\footnotesize
\begin{verbatim}
    SLIP_info SLIP_LU_analyze
    (
        SLIP_LU_analysis **S, // symbolic analysis (column permutation
                              // and nnz L,U)
        const SLIP_matrix *A, // Input matrix
        const SLIP_options *option  // Control parameters
    ) ;
\end{verbatim}
} \end{mdframed}

\verb|SLIP_LU_analyze| performs the symbolic ordering for SLIP LU. Currently,
there are three options: no ordering, COLAMD, or AMD, which are passed in by
\verb|SLIP_options| \verb|*option|. For more details, refer to
Section \ref{ss:SLIP_options}.

The \verb|SLIP_LU_analysis *S| is created by calling
\verb|SLIP_LU_analyze(&S, A, option)| with \verb|SLIP_matrix *A| properly
initialized as CSC matrix and \verb|option| be \verb|NULL| if default ordering
(COLAMD) is desired. The value of \verb|S| is ignored on input.  On output,
\verb|S| is a pointer to the newly created symbolic analysis object and
\verb|SLIP_OK| is returned upon successful completion, or \verb|S = NULL| with
error status returned if a failure occurred.  Returns \verb|SLIP_PANIC| if SLIP
LU has not been initialized.

The analysis \verb|S| is freed by \verb|SLIP_LU_analysis_free|.

%-------------------------------------------------------------------------------
\cprotect\subsection{\verb|SLIP_LU_analysis_free|: free \verb|SLIP_LU_analysis| structure}
\label{ss:LU_analysis_free}
%-------------------------------------------------------------------------------

\begin{mdframed}[userdefinedwidth=6in]
{\footnotesize
\begin{verbatim}
    SLIP_info SLIP_LU_analysis_free
    (
        SLIP_LU_analysis **S, // Structure to be deleted
        const SLIP_options *option
    ) ;
\end{verbatim}
} \end{mdframed}


\verb|SLIP_LU_analysis_free| frees a \verb|SLIP_LU_analysis| structure.
Note that the input of the function is the pointer to the pointer of a
\verb|SLIP_LU_analysis| structure. This is because this function internally
sets the pointer of a \verb|SLIP_LU_analysis| to be \verb|NULL| to prevent
potential segmentation fault that could be caused by double \verb|free|.
If default settings are desired, \verb|option| can be input as \verb|NULL|.
Returns \verb|SLIP_PANIC| if SLIP LU has not been initialized.

%-------------------------------------------------------------------------------
\cprotect\subsection{\verb|SLIP_LU_factorize|: perform LU factorization}
\label{ss:SLIP_LU_factorize}
%-------------------------------------------------------------------------------

\begin{mdframed}[userdefinedwidth=6in]
{\footnotesize
\begin{verbatim}
    SLIP_info SLIP_LU_factorize
    (
        // output:
        SLIP_matrix **L_handle,     // lower triangular matrix
        SLIP_matrix **U_handle,     // upper triangular matrix
        SLIP_matrix **rhos_handle,  // sequence of pivots
        int64_t **pinv_handle,      // inverse row permutation
        // input:
        const SLIP_matrix *A,        // matrix to be factored
        const SLIP_LU_analysis *S,   // column permutation and estimates
                                     // of nnz in L and U 
        const SLIP_options* option
    ) ;
\end{verbatim}
} \end{mdframed}

\verb|SLIP_LU_factorize| performs the SLIP LU factorization. This factorization
is done via $n$ (number of rows or columns of the square matrix $A$) iterations
of the sparse REF triangular solve function. The overall factorization is $PAQ
= LDU$.  This routine allows the factorization and solve to be split into
separate phases.  For example codes, refer to either
\verb|SLIP_LU/Demos/SLIPLU.c| or Section \ref{s:Using:expert}.

On input, \verb|L|, \verb|U|, \verb|rhos|, and \verb|pinv| are undefined and
ignored.  \verb|A| must be a CSC \verb|mpz| matrix. Default settings are used
if \verb|option| is input as \verb|NULL|.

Upon successful completion, the function returns \verb|SLIP_OK|, and \verb|L|
and \verb|U| are lower and upper triangular matrices, respectively, which are
CSC matrices of type \verb|mpz|.  \verb|rhos| contains the sequence of pivots
as an \verb|n|-by-1 dense vector of type \verb|mpz|.

After factorizing the matrix, the determinant of $A$ can be obtained from
\verb|rhos[n-1]| and \verb|A->scale| as follows:

\begin{verbatim}
    mpq_t determinant ;
    SLIP_mpq_init (determinant) ;
    SLIP_mpq_set_z (determinant, rhos->x.mpz[rhos->n-1]) ;
    SLIP_mpq_div (determinant, determinant, A->scale) ;
\end{verbatim}

The output array \verb|pinv| contains the inverse row permutation (that is, the
row index in the permuted matrix $PA$. For the $i$th row in $A$, \verb|pinv[i]|
gives the row index in $PA$). 

Returns \verb|SLIP_PANIC| if SLIP LU has not been initialized.  Otherwise, if
another error occurs, \verb|L|, \verb|U|, \verb|rhos|, and \verb|pinv| are all
returned as \verb|NULL|, and an error code will be returned correspondingly.

%-------------------------------------------------------------------------------
\newpage
\cprotect\subsection{\verb|SLIP_LU_solve|: solve the linear system $Ax=b$}
\label{ss:SLIP_LU_solve}
%-------------------------------------------------------------------------------

\begin{mdframed}[userdefinedwidth=6in]
{\footnotesize
\begin{verbatim}
    SLIP_info SLIP_LU_solve         // solves the linear system LD^(-1)U x = b
    (
        // Output
        SLIP_matrix **X_handle,     // rational solution to the system
        // input:
        const SLIP_matrix *b,       // right hand side vector
        const SLIP_matrix *A,       // Input matrix
        const SLIP_matrix *L,       // lower triangular matrix
        const SLIP_matrix *U,       // upper triangular matrix
        const SLIP_matrix *rhos,    // sequence of pivots
        const SLIP_LU_analysis *S,  // symbolic analysis struct
        const int64_t *pinv,        // inverse row permutation
        const SLIP_options* option
    ) ;
\end{verbatim}
} \end{mdframed}

\verb|SLIP_LU_solve| obtains the solution of \verb|mpq_t| type to the linear
system $Ax=b$ upon a successful factorization.  This function may be called
after a successful return from \verb|SLIP_LU_factorize|, which computes
\verb|L|, \verb|U|, \verb|rhos|, and \verb|pinv|. 

On input, \verb|SLIP_matrix *x| is undefined. \verb|A|, \verb|L| and \verb|U|
must be CSC \verb|mpz_t| matrices while \verb|b| and \verb|rhos| must be dense
\verb|mpz_t|  matrices. All matrices must have matched dimensions: the matrices
\verb|L| and \verb|U| must be square lower and upper triangular matrices the
same size as \verb|A|, and \verb|rhos| must be a dense \verb|n|-by-1 vector.
The input matrix \verb|b| must have same number of rows as \verb|A|.  Default
settings are used if \verb|option| is input as \verb|NULL|.

Upon successful completion, the function returns \verb|SLIP_OK|, and \verb|x|
contains the solution of \verb|mpq_t| type with dense format to the linear
system $Ax=b$. If desired, \verb|option->check| can be set to \verb|true| to
enable a post-check of the solution of this function.  However, this is
intended for debugging only; the SLIP LU library is guaranteed to return the
exact solution. Otherwise (in case of error occurred), the function returns
corresponding error code.

This function is primarily for applications that require intermediate results.
For additional information, refer to either \verb|SLIP_LU/Demos/SLIPLU.c| or
Section \ref{s:Using:expert}.  Returns \verb|SLIP_PANIC| if SLIP LU has not
been initialized.

%-------------------------------------------------------------------------------
\newpage
\cprotect\subsection{\verb|SLIP_backslash|: solve $Ax=b$}
\label{ss:SLIP_backslash}
%-------------------------------------------------------------------------------

\begin{mdframed}[userdefinedwidth=6in]
{\footnotesize
\begin{verbatim}
    SLIP_info SLIP_backslash
    (
        // Output
        SLIP_matrix **X_handle,       // Final solution vector
        // Input
        SLIP_type type,               // Type of output desired:
                                      // Must be SLIP_MPQ, SLIP_MPFR,
                                      // or SLIP_FP64
        const SLIP_matrix *A,         // Input matrix
        const SLIP_matrix *b,         // Right hand side vector(s)
        const SLIP_options* option
    ) ;
\end{verbatim}
} \end{mdframed}

\verb|SLIP_backslash| solves the linear system $Ax=b$ and returns the solution
as a dense matrix of \verb|mpq_t|, \verb|mpfr_t| or \verb|double| numbers. This
function performs symbolic analysis, factorization, and solving all in one line. 
It can be thought of as an exact version of MATLAB sparse backslash.

On input, \verb|SLIP_matrix *x| is undefined. \verb|type| must be one of:
\verb|SLIP_MPQ|, \verb|SLIP_MPFR| or \verb|SLIP_FP64| to specify the data type
of the solution entries. \verb|A| should be a square CSC \verb|mpz_t| matrix
while \verb|b| should be a dense \verb|mpz_t| matrix. In addition, \verb|A->m|
should be equal to \verb|b->m|.  Default settings are used if
\verb|option| is input as \verb|NULL|.

Upon successful completion, the function returns \verb|SLIP_OK|, and
\verb|x| contains the solution of data type specified by
\verb|type| to the linear system $Ax=b$. If desired, \verb|option->check| can
be set to \verb|true| to enable solution checking process in this function.
However, this is intended for debugging only; SLIP LU library is guaranteed to
return the exact solution. Otherwise (in case of error occurred), the function
returns corresponding error code.

Returns \verb|SLIP_PANIC| if SLIP LU has not been initialized.

For a complete example, refer to \verb|SLIP_LU/Demos/example.c|,  \\
\verb|SLIP_LU/Demos/example2.c|, or Section \ref{s:Using:simple}.

%-------------------------------------------------------------------------------
\section{SLIP LU wrapper functions for GMP and MPFR}
%-------------------------------------------------------------------------------

SLIP LU provides a wrapper class for all GMP and MPFR functions used by SLIP
LU.  The wrapper class provides error-handling for out-of-memory conditions
that are not handled by the GMP and MPFR libraries.  These wrapper functions
are used inside all SLIP LU functions, wherever any GMP or MPFR functions are
used.  These functions may also be called by the end-user application.

Each wrapped function has the same name as its corresponding GMP/MPFR function
with the added prefix \verb|SLIP_|. For example, the default GMP function
\verb|mpz_mul| is changed to \verb|SLIP_mpz_mul|. Each SLIP GMP/MPFR function
returns \verb|SLIP_OK| if successful or the correct error code if not. The
following table gives a brief list of each currently covered SLIP GMP/MPFR
function. For a detailed description of each function, refer to
\verb|SLIP_LU/Source/SLIP_gmp.c|.

If additional GMP and MPFR functions are needed in the end-user application,
this wrapper mechanism can be extended to those functions.  Below are
instructions on how to do this.

Given a GMP function \verb|void gmpfunc(TYPEa a, TYPEb b, ...)|, where
\verb|TYPEa| and \verb|TYPEb| can be GMP type data (\verb|mpz_t|,
\verb|mpq_t| and \verb|mpfr_t|, for example) or non-GMP type data (\verb|int|,
\verb|double|, for example), and they need not to be the same.
A wrapper for a new GMP or MPFR function can be created by following
this outline:

\begin{mdframed}[userdefinedwidth=6in]
{\footnotesize
\begin{verbatim}
SLIP_info SLIP_gmpfunc
(
    TYPEa a,
    TYPEb b,
    ...
)
{
    // Start the GMP Wrappter
    // uncomment one of the following:
    // If this function is not modifying any GMP/MPFR type variable, then use
    //SLIP_GMP_WRAPPER_START;
    // If this function is modifying mpz_t type (say TYPEa = mpz_t), then use
    //SLIP_GMPZ_WRAPPER_START(a) ;
    // If this function is modifying mpq_t type (say TYPEa = mpq_t), then use
    //SLIP_GMPQ_WRAPPER_START(a) ;
    // If this function is modifying mpfr_t type (say TYPEa = mpfr_t), then use
    //SLIP_GMPFR_WRAPPER_START(a) ;

    // Call the GMP function
    gmpfunc(a,b,...) ;

    //Finish the wrapper and return ok if successful.
    SLIP_GMP_WRAPPER_FINISH;
    return SLIP_OK;
}
\end{verbatim}
} \end{mdframed}

Note that, other than \verb|SLIP_mpfr_fprintf|, \verb|SLIP_gmp_fprintf|,
\verb|SLIP_gmp_printf| and \verb|SLIP_gmp_fscanf|, all of the wrapped GMP/MPFR
functions always return \verb|SLIP_info| to the caller. Therefore, for some
GMP/MPFR functions that have their own return value.  For example, for
\verb|int mpq_cmp(const mpq_t a, const mpq_t b)|, the return value becomes a
parameter of the wrapped function. In general, a GMP/MPFR function in the form
of \verb|TYPEr gmpfunc(TYPEa a, TYPEb b, ...)|, the wrapped
function can be constructed as follows:

\begin{mdframed}[userdefinedwidth=6in]
{\footnotesize
\begin{verbatim}
SLIP_info SLIP_gmpfunc
(
    TYPEr *r,        // return value of the GMP/MPFR function
    TYPEa a,
    TYPEb b,
    ...
)
{
    // Start the GMP Wrappter
    //SLIP_GMP_WRAPPER_START;

    // Call the GMP function
    *r = gmpfunc(a,b,...) ;

    //Finish the wrapper and return ok if successful.
    SLIP_GMP_WRAPPER_FINISH;
    return SLIP_OK;
}
\end{verbatim}
} \end{mdframed}

% \newpage
\thispagestyle{empty}
{\scriptsize
\begin{center}
\begin{tabular}{|l|l|l|}
\hline
%----------------------------------------
{\bf MPFR Function} & \verb|SLIP_MPFR| {\bf Function} & {\bf Description} \\
%----------------------------------------
\hline\hline
\verb|n = mpfr_asprintf(&buff, fmt, ...)|
    & \verb|n = SLIP_mpfr_asprintf(&buff, fmt, ...)|
    & Print format to allocated string \\ \hline
\verb|mpfr_free_str(buff)|
    & \verb|SLIP_mpfr_free_str(buff)|
    & Free string allocated by MPFR \\ \hline
\verb|mpfr_init2(x, size)|
    & \verb|SLIP_mpfr_init2(x, size)|
    & Initialize x with size bits \\ \hline
\verb|mpfr_set(x, y, rnd)|
    & \verb|SLIP_mpfr_set(x, y, rnd)|
    & $x = y$ \\ \hline
\verb|mpfr_set_d(x, y, rnd)|
    & \verb|SLIP_mpfr_set_d(x, y, rnd)|
    & $x = y$ (double) \\ \hline
\verb|mpfr_set_q(x, y, rnd)|
    & \verb|SLIP_mpfr_set_q(x, y, rnd)|
    & $x = y$ (\verb|mpq_t|) \\ \hline
\verb|mpfr_set_z(x, y, rnd)|
    & \verb|SLIP_mpfr_set_z(x, y, rnd)|
    & $x = y$ (\verb|mpz_t|) \\ \hline
\verb|mpfr_get_z(x, y, rnd)|
    & \verb|SLIP_mpfr_get_z(x, y, rnd)|
    & (\verb|mpz_t|) $x = y$\\ \hline
\verb|x = mpfr_get_d(y, rnd)|
    & \verb|SLIP_mpfr_get_d(x, y, rnd)|
    & (double) $x = y$\\ \hline
\verb|mpfr_mul(x, y, z, rnd)|
    & \verb|SLIP_mpfr_mul(x, y, z, rnd)|
    & $x = y*z$ \\ \hline
\verb|mpfr_mul_d(x, y, z, rnd)|
    & \verb|SLIP_mpfr_mul_d(x, y, z, rnd)|
    & $x = y*z$ \\ \hline
\verb|mpfr_div_d(x, y, z, rnd)|
    & \verb|SLIP_mpfr_div_d(x, y, z, rnd)|
    & $x = y/z$ \\ \hline
\verb|mpfr_ui_pow_ui(x, y, z, rnd)|
    & \verb|SLIP_mpfr_ui_pow_ui(x, y, z, rnd)|
    & $x = y^z$ \\ \hline
\verb|mpfr_log2(x, y, rnd)|
    & \verb|SLIP_mpfr_log2(x, y, rnd )|
    & $x = \log_2 (y)$ \\ \hline
\verb|mpfr_free_cache()|
    & \verb|SLIP_mpfr_free_cache()|
    & Free cache after log2 \\ \hline
\hline
%----------------------------------------
{\bf GMP Function} & \verb|SLIP_GMP| {\bf Function} & {\bf Description} \\
%----------------------------------------
\hline\hline
\verb|n = gmp_fscanf(fp, fmt, ...)|
    & \verb|n = SLIP_gmp_fscanf(fp, fmt, ...)|
    & Read from file fp \\ \hline
\verb|mpz_init(x)|
    & \verb|SLIP_mpz_init(x)|
    & Initialize x \\ \hline
\verb|mpz_init2(x, size)|
    & \verb|SLIP_mpz_init2(x, size)|
    & Initialize x to size bits \\ \hline
\verb|mpz_set(x, y)|
    & \verb|SLIP_mpz_set(x, y)| 
    & $x = y$ (\verb|mpz_t|) \\ \hline
\verb|mpz_set_ui(x, y)|
    & \verb|SLIP_mpz_set_ui(x, y)|
    & $x = y$ (signed int) \\ \hline
\verb|mpz_set_si(x, y)|
    & \verb|SLIP_mpz_set_si(x, y)|
    & $x = y$ (unsigned int) \\ \hline
\verb|mpz_set_d(x, y)|
    & \verb|SLIP_mpz_set_d(x, y)|
    & $x = y$ (double)\\ \hline
\verb|x = mpz_get_d(y)|
    & \verb|SLIP_mpz_get_d(x, y)|
    & $x = y$ (double out) \\ \hline
\verb|mpz_set_q(x, y)|
    & \verb|SLIP_mpz_set_q(x, y)|
    & $x = y$ (\verb|mpz_t|) \\ \hline
\verb|mpz_mul(x, y, z)|
    & \verb|SLIP_mpz_mul(x, y, z)|
    & $x = y*z$ \\ \hline
\verb|mpz_add(x, y, z)|
    & \verb|SLIP_mpz_add(x, y, z)|
    & $x = y+z$ \\ \hline
\verb|mpz_addmul(x, y, z)|
    & \verb|SLIP_mpz_addmul(x, y, z)|
    & $x = x+y*z$ \\ \hline
\verb|mpz_submul(x, y, z)|
    & \verb|SLIP_mpz_submul(x, y, z)|
    & $x = x-y*z$ \\ \hline
\verb|mpz_divexact(x, y, z)|
    & \verb|SLIP_mpz_divexact(x, y, z)|
    & $x = y/z$ \\ \hline
\verb|gcd = mpz_gcd(x, y)|
    & \verb|SLIP_mpz_gcd(gcd, x, y)|
    & $gcd = gcd(x,y)$\\ \hline
\verb|lcm = mpz_lcm(x, y)|
    & \verb|SLIP_mpz_lcm(lcm, x, y)|
    & $lcm = lcm(x,y)$ \\ \hline
\verb|mpz_abs(x, y)|
    & \verb|SLIP_mpz_abs(x, y)|
    & $x = |y|$ \\ \hline
\verb|r = mpz_cmp(x, y)|
    & \verb|SLIP_mpz_cmp(r, x, y)|
    & $r = 0$ if $x=y$, $r\neq 0$  if $x\neq y$ \\ \hline
\verb|r = mpz_cmpabs(x, y)|
    & \verb|SLIP_mpz_cmpabs(r, x, y)|
    & $r = 0$ if $|x|=|y|$,  $r\neq 0$  if $|x|\neq |y|$\\ \hline
\verb|r = mpz_cmp_ui(x, y)|
    & \verb|SLIP_mpz_cmp_ui(r, x, y)|
    & $r = 0$ if $x=y$,  $r\neq 0$  if $x\neq y$ \\ \hline
\verb|sgn = mpz_sgn(x)|
    & \verb|SLIP_mpz_sgn(sgn, x)|
    & $sgn = 0$ if $x = 0$ \\ \hline
\verb|size = mpz_sizeinbase(x, base)|
    & \verb|SLIP_mpz_sizeinbase(size, x, base)|
    & size of x in base \\ \hline
\verb|mpq_init(x)|
    & \verb|SLIP_mpq_init(x)|
    & Initialize x \\ \hline
\verb|mpq_set(x, y)|
    & \verb|SLIP_mpq_set(x, y)|
    & $x = y$ \\ \hline
\verb|mpq_set_z(x, y)|
    & \verb|SLIP_mpq_set_z(x, y)|
    & $x = y$ (\verb|mpz|) \\ \hline
\verb|mpq_set_d(x, y)|
    & \verb|SLIP_mpq_set_d(x, y)|
    & $x=y$ (double) \\ \hline
\verb|mpq_set_ui(x, y, z)|
    & \verb|SLIP_mpq_set_ui(x, y, z)|
    & $x = y/z$ (unsigned int) \\ \hline
\verb|mpq_set_num(x, y)|
    & \verb|SLIP_mpq_set_num(x, y)|
    & $num(x) = y$ \\ \hline
\verb|mpq_set_den(x, y)|
    & \verb|SLIP_mpq_set_den(x, y)|
    & $den(x) = y$ \\ \hline
\verb|mpq_get_den(x, y)|
    & \verb|SLIP_mpq_get_den(x, y)|
    & $x = den(y)$ \\ \hline
\verb|x = mpq_get_d(y)|
    & \verb|SLIP_mpq_get_d(x, y)|
    & (double) $x = y$ \\ \hline
\verb|mpq_abs(x, y)|
    & \verb|SLIP_mpq_abs(x, y)|
    & $x = |y|$ \\ \hline
\verb|mpq_add(x, y, z)|
    & \verb|SLIP_mpq_add(x, y, z)|
    & $x = y+z$ \\ \hline
\verb|mpq_mul(x, y, z)|
    & \verb|SLIP_mpq_mul(x, y, z)|
    & $x = y*z$ \\ \hline
\verb|mpq_div(x, y, z)|
    & \verb|SLIP_mpq_div(x, y, z)|
    & $x = y/z$ \\ \hline
\verb|r = mpq_cmp(x, y)|
    & \verb|SLIP_mpq_cmp(r, x, y)|
    & $r = 0$ if $x=y$,  $r\neq 0$ if $x\neq y$ \\ \hline
\verb|r = mpq_cmp_ui(x, n, d)|
    & \verb|SLIP_mpq_cmp_ui(r, x, n, d)|
    & $r = 0$ if $x=n/d$, $r\neq 0$ if $x\neq n/d$ \\ \hline
\verb|r = mpq_equal(x, y)|
    & \verb|SLIP_mpq_equal(r, x, y)|
    & $r = 0$ if $x=y$,  $r\neq 0$ if $x\neq y$ \\ \hline
\end{tabular}
\end{center}
}

%-------------------------------------------------------------------------------
\cprotect\section{Using SLIP LU in C} \label{s:Using}
%-------------------------------------------------------------------------------

Using SLIP LU in C has three steps:

\begin{enumerate}
\item initialize and populate data structures,
\item perform symbolic analysis,
factorize the matrix $A$ and solve the linear
system for each $b$ vector, and
\item free all used memory and finalize.
\end{enumerate}

Step 1 is discussed in Section \ref{s:Using:init}.  For Step 2, performing
symbolic analysis and factorizing $A$ and solving the linear $A x =b$ can be
done in one of two ways. If only the solution vector $x$ is required, SLIP LU
provides a simple interface for this purpose which is discussed in Section
\ref{s:Using:simple}.  Alternatively, if the $L$ and $U$ factors are required,
refer to Section \ref{s:Using:expert}.  Finally, step 3 is discussed in Section
\ref{s:Using:free}. For the remainder of this section, \verb|n| will indicate
the dimension of $A$ (that is, $A \in \mathbb{Z}^{n \times n}$) and
\verb|numRHS| will indicate the number of right hand side vectors being solved
(that is, if \verb|numRHS|$= r$, then $b \in \mathbb{Z}^{n \times r}$).

%-------------------------------------------------------------------------------
\cprotect\subsection{SLIP LU initialization and population of data structures}
\label{s:Using:init}
%-------------------------------------------------------------------------------

This section discusses how to initialize and populate the global data
structures required for SLIP LU.

%-------------------------------------------------------------------------------
\subsubsection{Initializing the environment}
%-------------------------------------------------------------------------------

SLIP LU is built upon the GNU Multiple Precision Arithmetic (GMP)
\cite{granlund2015gnu} and GNU Multiple Precision Floating Point Reliable
(MPFR) \cite{fousse2007mpfr} libraries and provides wrappers to all GMP/MPFR
functions it uses.  This allows SLIP LU to properly handle memory management
failures, which GMP/MPFR does not handle.  To enable this mechanism, SLIP LU
requires initialization.  The following must be done before using any other
SLIP LU function:

\begin{mdframed}[userdefinedwidth=6in]
{\footnotesize
\begin{verbatim}
    SLIP_initialize ( ) ;
    // or SLIP_initialize_expert (...); if custom memory functions are desired
\end{verbatim}
} \end{mdframed}

%-------------------------------------------------------------------------------
\subsubsection{Initializing data structures}
\label{ss:init}
%-------------------------------------------------------------------------------

SLIP LU assumes three specific input options for all functions. These are:

\begin{itemize}
\item \verb|SLIP_matrix* A| and \verb|SLIP_matrix *b|: \verb|A| contains the
input coefficient matrix, while \verb|b| contains the right hand side vector(s)
of the linear system $Ax=b$.

\item \verb|SLIP_LU_analysis* S|: \verb|S| contains the column permutation used
for $A$ as well as estimates of the number of nonzeros in $L$ and $U$.

\item \verb|SLIP_options* option|: \verb|option| contains various control
options for the factorization including column ordering used, pivot selection
scheme, and others. For a full list of the contents of the \verb|SLIP_options|
structure, refer to Section \ref{ss:SLIP_options}.
If default settings are desired, \verb|option| can be set to \verb|NULL|.

\end{itemize}

%-------------------------------------------------------------------------------
\subsubsection{Populating data structures}
\label{ss:populate_Ab}
%-------------------------------------------------------------------------------

Of the three data structures discussed in Section~\ref{ss:init}, \verb|S| is
constructed during symbolic analysis (Section \ref{s:SLIP_LU_analyze}), and
\verb|option| is an optional parameter for selecting non-default parameters.
Refer to Section \ref{ss:SLIP_options} for the contents of \verb|option|.

SLIP LU allows the input numerical data for \verb|A| and \verb|b| to come in
one of 5 types: \verb|int64_t|, \verb|double|, \verb|mpfr_t|, \verb|mpq_t|,
and \verb|mpz_t|. Moreover, both \verb|A| and \verb|b| can be stored in
CSC form, sparse triplet form or dense form. CSC form is discussed in Section
\ref{s:intro}. The triplet form stores the contents of the matrix $A$
in three arrays \verb|i|, \verb|j|, and \verb|x| where the $k$th nonzero entry
is stored as $A ( i[k], j[k]) = x[k]$. SLIP LU stores its dense matrices in
in column-oriented format, that is, the $(i,j)$th entry in \verb|A|
is \verb|A->x.TYPE[p]| with $p = i+j$*\verb|A->m|.

If the data for matrices are in file format to be read, refer to
\verb|SLIP_LU/Demo| \verb|/example2.c| on how to read in data and construct
\verb|A| and \verb|b|. If the data for matrices are already stored in vectors
corresponding to CSC form, sparse triplet form or dense form, allocate a
shallow \verb|SLIP_matrix| and assign vectors accordingly, then use
\verb|SLIP_matrix_copy| to get a \verb|SLIP_matrix| in the desired kind and
type. For more details, refer to \verb|SLIP_LU/Demo/example.c|. In a case when
\verb|A| is available in format other than CSC \verb|mpz|, and/or \verb|b| is
available in format other than dense \verb|mpz|, the following code snippet
shows how to get \verb|A| and \verb|b| in a required format.

{\small
\begin{verbatim}

    /* Get the matrix A. Assume that A1 is stored in CSC form
       with mpfr_t entries, while b1 is stored in triplet form
       with mpq_t entries. (for A1 and b1 in any other form,
       the exact same code will work) */

    SLIP_matrix *A, *b;
    // A is a copy of the A1. A is a CSC matrix with mpz_t entries
    SLIP_matrix_copy(&A, SLIP_CSC,   SLIP_MPZ, A1, option);
    // b is a copy of the b1. b is a dense matrix with mpz_t entries. 
    SLIP_matrix_copy(&b, SLIP_DENSE, SLIP_MPZ, b1, option);
    \end{verbatim} }

%-------------------------------------------------------------------------------
\cprotect\subsection{Simple SLIP LU routines for solving linear systems}
\label{s:Using:simple}
%-------------------------------------------------------------------------------

After initializing the necessary data structures, SLIP LU obtains the solution
to $Ax=b$ using the simple interface of SLIP LU, \verb|SLIP_backslash|.  The
\verb|SLIP_backslash| function can return \verb|x| as \verb|double|,
\verb|mpq_t|, or \verb|mpfr_t| with an associated precision.  See Section
\ref{ss:SLIP_backslash} for more details.  The following code snippet shows how
to get solution as a dense \verb|mpq_t| matrix.

{\small
\begin{verbatim}
    SLIP_matrix *x;
    SLIP_type my_type = SLIP_MPQ; // SLIP_MPQ, SLIP_MPFR, SLIP_FP64
    SLIP_backslash(&x, my_type, A, b, option) ; \end{verbatim} }

On successful return, this function returns \verb|SLIP_OK| (see Section
\ref{ss:SLIP_info}).

%-------------------------------------------------------------------------------
\cprotect\subsection{Expert SLIP LU routines}
\label{s:Using:expert}
%-------------------------------------------------------------------------------

If the $L$ and $U$ factors from the SLIP LU factorization of the matrix $A$
are required, the steps performed by \verb|SLIP_backslash| can be done with
a sequence of calls to SLIP LU functions:

\begin{enumerate}
\item declare \verb|L|, \verb|U|, the solution matrix \verb|x|, and others,
\item perform symbolic analysis,
\item compute the factorization $PAQ = L D U$, 
\item solve the linear system $Ax =b$, and
\item convert the final solution into the final desired form.
\end{enumerate}

These steps are discussed below, along with examples.

%-------------------------------------------------------------------------------
\subsubsection{Declare workspace}
%-------------------------------------------------------------------------------

Using SLIP LU in this form requires the intermediate variables be declared,
such as \verb|L|, \verb|U|, etc. The following code snippet shows the
detailed list.

{\small
\begin{verbatim}
    // A and b are in required type and ready to use
    SLIP_matrix *L = NULL;
    SLIP_matrix *U = NULL;
    SLIP_matrix *x = NULL;
    SLIP_matrix *rhos = NULL;
    int64_t* pinv = NULL;
    SLIP_LU_analysis* S = NULL;

    // option needs no declaration if default setting is desired
    // only declare option for further modification on default setting
    SLIP_options *option = SLIP_create_default_options();
     \end{verbatim} }

%-------------------------------------------------------------------------------
\subsubsection{SLIP LU symbolic analysis}
%-------------------------------------------------------------------------------

The symbolic analysis phase of SLIP LU computes the column permutation and
estimates of the number of nonzeros in $L$ and $U$. This function is called as:

{\small
    \begin{verbatim}
    SLIP_LU_analyze (&S, A, option) ; \end{verbatim} }


%-------------------------------------------------------------------------------
\subsubsection{Computing the factorization}
%-------------------------------------------------------------------------------

The matrices \verb|L| and \verb|U|, the pivot sequence \verb|rhos|, and the row
permutation \verb|pinv| are computed via the \verb|SLIP_LU_factorize| function
(Section \ref{ss:SLIP_LU_factorize}).  Upon successful completion, this
function returns \verb|SLIP_OK|.

%-------------------------------------------------------------------------------
\subsubsection{Solving the linear system}
%-------------------------------------------------------------------------------

After factorization, the next step is to solve the linear system and store the
solution as a dense matrix \verb|x| with entries of rational number
\verb|mpq_t|. This solution is done via the \verb|SLIP_LU_solve|
function (Section \ref{ss:SLIP_LU_solve}). 
Upon successful completion, this function returns \verb|SLIP_OK|.

In this step, \verb|option->check| can be set to \verb|true| to enable the
solution check process as discussed in Section \ref{ss:SLIP_LU_solve}.  The
process can verify that the solution vector x satisfies $Ax=b$ in perfect
precision intended for debugging.  This step is not needed, since the solution
returned is guaranteed to be exact.   It appears here simply as debugging tool,
and as a verification that SLIP LU is computing its expected result.  This test
can fail only if it runs out of memory, or if there is a bug in the code (in
which case, please notify the authors).  Also, note that this process can be
quite time consuming; thus it is not recommended to be used in general.

%-------------------------------------------------------------------------------
\subsubsection{Converting the solution vector to the final desired form}
%-------------------------------------------------------------------------------

Upon completion of the above routines, the solution to the linear system is in
a dense \verb|mpq_t| matrix. SLIP LU allows this to be converted into any form
of matrix in the set of (CSC, sparse triplet, dense) $\times$ (\verb|mpfr_t|,
\verb|mpq_t|, \verb|double|) using \verb|SLIP_matrix_copy|. The following code
snippet shows how to get solution as a dense \verb|double| matrix; since this
involves a floating-point representation, the solution \verb|my_x| will no
longer be exact, even though \verb|x| is the exact solution.

{\small
\begin{verbatim}
    SLIP_kind my_kind = SLIP_DENSE;  // SLIP_CSC, SLIP_TRIPLET or SLIP_DENSE
    SLIP_type my_type = SLIP_FP64;   // SLIP_MPQ, SLIP_MPFR, or SLIP_FP64
    SLIP_matrix* my_x = NULL;        // New output
    // Create copy which is stored as my_kind and my_type:
    SLIP_matrix_copy( &my_x, my_kind, my_type, x, option);\end{verbatim} }

%-------------------------------------------------------------------------------
\cprotect\subsection{Freeing memory}
\label{s:Using:free}
%-------------------------------------------------------------------------------

As described in Section \ref{s:user:memmanag}, SLIP LU provides a number
of functions/macros to free SLIP LU objects:

\begin{itemize}
\item \verb|SLIP_matrix*|: A \verb|SLIP_matrix* A| data structure can be freed
with a call to \verb|SLIP_matrix_free(&A, NULL) ;|

\item \verb|SLIP_LU_analysis*|: A \verb|SLIP_LU_analysis* S| data structure can
be freed with a call to \verb|SLIP_LU_analysis_free(&S, NULL) ;|

\item All others including \verb|SLIP_options*|: These data structures can be
freed with a call to the macro \verb|SLIP_FREE()|, for example,
\verb|SLIP_FREE(option)| for \newline
\verb|SLIP_options* option|.

\end{itemize}

After all usage of the SLIP LU routines is finished, \verb|SLIP_finalize()|
must be called (Section \ref{ss:SLIP_finalize}) to finalize usage of the
library.

%-------------------------------------------------------------------------------
\cprotect\subsection{Examples of using SLIP LU in a C program}
\label{s:Using:Examples}
%-------------------------------------------------------------------------------

The \verb|SLIP_LU/Demo| folder contains three sample C codes which utilize SLIP
LU. These files demonstrate the usage of SLIP LU as follows:

\begin{itemize}
\item \verb|example.c|: This example generates a random dense $50 \times 50$
matrix and a random dense $50 \times 1$ right hand side vector $b$ and
solves the linear system. In this function, the \verb|SLIP_backslash|
function is used; and the output is given as a double matrix.

\item \verb|example2.c|: This example reads in a matrix stored in triplet
format from the \verb|ExampleMats| folder. Additionally, it reads in a
right hand side vector from this folder and solves the associated linear system
via the \verb|SLIP_backslash| function, and, the solution is given as a matrix
of rational numbers.

\item \verb|SLIPLU.c|: This example reads in a matrix and right hand side
vector from a file and solves the linear system $A x = b$
using the techniques discussed in Section \ref{s:Using:expert}. This file also
allows command line arguments (discussed in \verb|README.md|) and can be used
to replicate the results from \cite{lourenco2019exact}.

\end{itemize}

%-------------------------------------------------------------------------------
\newpage
\cprotect\section{Using SLIP LU in MATLAB}
\label{s:Use:MATLAB}
%-------------------------------------------------------------------------------

After following the installation steps discussed in Section \ref{s:install},
using the SLIP LU factorization within MATLAB can be done via the
\verb|SLIP_backslash.m| function. First, this section describes the
\verb|option| struct in Section \ref{s:Use:MATLAB:setup}.
The use of the factorization is discussed in Section \ref{s:Use:MATLAB:factor}.
The \verb|SLIP_LU/MATLAB| folder must be in your MATLAB path.

%-------------------------------------------------------------------------------
\cprotect\subsection{Optional parameter settings}
\label{s:Use:MATLAB:setup}
%-------------------------------------------------------------------------------

The SLIP LU MATLAB interface includes an \verb|option| struct as in optional
input parameter that modifies behavior.  If this parameter is not provided,
default parameter settings are used.  The elements of the \verb'option' struct
are listed below.  Any fields not present in the struct are treated as their
default values.

\begin{itemize}

\item \verb|option.pivot|: This parameter is a string that controls the
pivoting scheme used.  When selecting a pivot entry in a given column, the
factorization method uses one of the following pivoting strategies:

    \begin{itemize}
    \item \verb|'smallest'|: smallest pivot,
    \item \verb|'diagonal'|: diagonal pivot if possible, otherwise smallest pivot,
    \item \verb|'first'|: first nonzero pivot in each column,
    \item \verb|'tol smallest'|: (default) diagonal pivot with a tolerance (\verb|option.tol|)
        for the smallest pivot,
    \item \verb|'tol largest'|: diagonal pivot with a tolerance (\verb|option.tol|)
        for the largest pivot,
    \item \verb|'largest'|: largest pivot.
    \end{itemize}
    
\item \verb|option.order|: This parameter is a string controls the
fill-reducing column preordering used.

    \begin{itemize}
    \item \verb|'none'|: no column ordering; factorize \verb'A' as-is.
    \item \verb|'colamd'|: COLAMD ordering (default)
    \item \verb|'amd'|: AMD ordering
    \end{itemize}

The \verb|'colamd'| is recommended for most cases.  The \verb|'AMD'| ordering
is suitable if the nonzero pattern of \verb'A' is mostly symmetric.  In this
case, \verb|option.pivot = 'diagonal'| is a useful option.

\item \verb|option.tol|: This parameter determines the tolerance used if one of
the threshold pivoting schemes is chosen. The default value is 1 and this
parameter can take any value in the range $(0,1]$.

\item \verb|option.solution|:
    a string determining how \verb|x| is to be returned:

    \begin{itemize}
        \item \verb|'double'|:  \verb|x| is converted to a 64-bit
            floating-point approximate solution.  This is the default.
        \item \verb|'vpa'|:  \verb|x| is returned as a \verb|vpa| array with
            \verb|option.digits| digits (default is given by the MATLAB
            \verb|digits| function).  The result may be inexact, if an entry in
            \verb|x| cannot be represented in the specified number of digits.
            To convert this \verb|x| to double, use \verb|x=double(x)|.
        \item \verb|'char'|:  \verb|x| is returned as a cell array of strings,
            where \verb|x {i} =| \newline \verb|'numerator/denominator'| and both
            \verb|numerator| and \verb|denominator| are arbitrary-length
            strings of decimal digits.  The result is always exact, although
            \verb|x| cannot be directly used in MATLAB for numerical
            calculations.  It can be inspected or analyzed using MATLAB string
            manipulation.  To convert \verb|x| to \verb|vpa|, use
            \verb|x=vpa(x)|.  To convert \verb|x| to double, use
            \verb|x=double(vpa(x))|.
    \end{itemize}

\item \verb|option.digits|: the number of decimal digits to use for \verb|x|, if
        \verb|option.solution| is \verb|'vpa'|.  Must be in range 2 to $2^{29}$.

\item \verb|option.print|: display the inputs and outputs
        (0: nothing (default), 1: just errors, 2: terse, 3: all).

\end{itemize}

%-------------------------------------------------------------------------------
\cprotect\subsection{\verb|SLIP_backslash.m|}
\label{s:Use:MATLAB:factor}
%-------------------------------------------------------------------------------

The \verb|SLIP_backslash.m| function solves the linear system $A x = b$ where
$A \in \mathtt{R}^{n \times n}$, $x \in \mathtt{R}^{n \times m}$ and $b \in
\mathtt{R}^{n \times m}$. The final solution vector(s) obtained via this
function are exact prior to their conversion to double precision.

The SLIP LU function expects as input a sparse matrix $A$ and dense set of
right hand side vectors $b$. Optionally, \verb|option| struct can be passed in.
Currently, there are 2 ways to use this function outlined below:

\begin{itemize}

\item \verb|x = SLIP_backslash(A,b)| returns the solution to $A x =
b$ using default settings. The solution vectors are more accurate than
the solution obtained via \verb|x = A \ b|.  The solution \verb|x| is
returned as a MATLAB double matrix.

\item \verb|x = SLIP_backslash(A,b,option)| returns the solution to $A x =
b$ using non-default settings from the \verb|option| struct.

\end{itemize}

If the result \verb|x| is held as a MATLAB double matrix, in conventional
floating-point representation (\verb|double|), it is guaranteed to be exact
only if the exact solution can be held in \verb|double| without modification.

The solution \verb|x| may also be returned as a MATLAB \verb|vpa| array, or as
a cell array of strings; See Section \ref{s:Use:MATLAB:setup} for details.

%-------------------------------------------------------------------------------
% References
%-------------------------------------------------------------------------------

\newpage
\bibliographystyle{siam}
\bibliography{SLIP_LU_UserGuide.bib}
\end{document}

